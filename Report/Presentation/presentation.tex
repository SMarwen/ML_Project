\documentclass[10pt,a4paper,oneside]{article}
\usepackage[utf8]{inputenc}
\usepackage{geometry}
 \geometry{
 a4paper,
 total={170mm,257mm},
 left=20mm,
 top=20mm,
 }
 \newcommand\independent{\protect\mathpalette{\protect\independenT}{\perp}}
\def\independenT#1#2{\mathrel{\rlap{$#1#2$}\mkern2mu{#1#2}}}
\usepackage{amsmath}
\usepackage{amsfonts}
\usepackage{amssymb}
\usepackage{graphicx}
\usepackage{color}


\author{Mohamed-El-Amine Seddik \hspace*{2em} Marwen Sallem \hspace*{2em} Kais Slimi \hspace*{2em} Oussama Bouraoui}
\title{ML Project : Machine Learning Methods Applied to Human Physical Activity Classification Using On-Body Sensors}
\begin{document}
\maketitle
\begin{abstract}
This paper describes our choice for the machine learning project subject, we chose to work on applying machine learning methods to the problem of human physical activity classification, based on data that comes from on-body sensors. Basically, the aim of this project is to explore different techniques of machine learning and to apply them to this particular problem.
\end{abstract}
\section*{Motivation}
The ability of designing a system that automatically classifies the physical activity performed by a given person is very attractive for many applications in various fields such that health-care monitoring and in developing smart human-machine interfaces. There are several techniques to handle this task, and mainly some of them are based on computer vision where some others aims to use on-body sensors. Based on [1] we chose to work on data that comes from on-body sensors. In fact, there is many datasets available on the internet that are used for this special problem, we provide below a link to a dataset that we plan to use for our project. 
\section*{Goal}
As illustrated by the title, our main goal is to classify the Human Physical Activity. To reach this goal we plan to explore different machine learning methods (Probabilistic based methods like HMM and geometric based ones as SVM k-NN ...) and to apply them to this particular task, at the end we aim to provide the performances of the different applied methods.
\section*{Dataset}
- https://archive.ics.uci.edu/ml/datasets/Activity+Recognition+from+Single+Chest-Mounted+Accelerometer


\section*{References}
[1] Mannini, A. Sabatini, A.M.	Machine Learning Methods for Classifying Human Physical Activity from On-Body Accelerometers. Sensors 2010, 10, 1154-1175.

\end{document}